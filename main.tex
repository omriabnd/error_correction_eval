\documentclass[english]{article}
\usepackage[T1]{fontenc}
\usepackage[latin9]{inputenc}
\usepackage{amssymb}
\usepackage{graphicx}

\makeatletter
%%%%%%%%%%%%%%%%%%%%%%%%%%%%%% Textclass specific LaTeX commands.
\newcommand{\lyxaddress}[1]{
\par {\raggedright #1
\vspace{1.4em}
\noindent\par}
}
\newcommand{\lyxrightaddress}[1]{
\par {\raggedleft \begin{tabular}{l}\ignorespaces
#1
\end{tabular}
\vspace{1.4em}
\par}
}

%%%%%%%%%%%%%%%%%%%%%%%%%%%%%% User specified LaTeX commands.
\usepackage{acl2015}

\usepackage{babel}
\begin{document}

\title{Improved Statistical Evaluation for Grammatical Error Correction}

\author{
  Leshem Choshen\textsuperscript{1} and Omri Abend\textsuperscript{2} \\
  \textsuperscript{1}School of Computer Science and Engineering, \textsuperscript{2} Department of Cognitive Sciences \\
  The Hebrew University of Jerusalem \\
  \texttt{leshem.choshen@mail.huji.ac.il, oabend@cs.huji.ac.il}\\
}


%%% It's double-blind, so no need for authors for the time being

\maketitle


\section{Main ideas(abstract base)}

We propose two notiong of conservatism:semantic conservatism and formal
conservatism and state the former one is mostly the one we strive
for when correcting.

We state semantic annotation is valuable for error correction and
error assessment.

We present a new method to compare UCCA annotation

we show UCCA can be used for annotating ungrammatical texts

we show UCCA is stable over grammar correction

we show state of the art error correction methods are highly firmally
conservative (why formal and not structural or another name?)

We show current correction methods are too formally conservative,
they don't change enough sentence boundaries, enough words and enough
charachters.

We suggest one of the factors contributing to over formal conservatism
is having only a small number of references covering low percentage
of the possible correction. It affects both assessment and development.
In the development process we would expect good learning to understand
not to try to correct complicated sentences as those will be very
likely to be judged a mistake. Even when they are not.

Current correctors undercorrect, maybe due to lack of gold references.

Current assessment is a an under assessment giving a much lower score
than ought to be.

Significance of current scores, and what can we say about it...

\section{Needed background}

learner language is hypothesized to be a consistent language allowing
us to think of error correction as a translation for closely related
languages.

ungrammatical texts are being a main subject to research where learner
language can be seen as one case.

\section{Opening}

In the history of error correction conservatism was considered an
important trait of an error correcting system\cite{brockett2006correcting}.
This was also the reason why $F_{0.5}$ became since conll2014\cite{ng2014conll}
the measure of choice for error correction evaluation, emphasizing
precision over recall. This emphasis can be understood as encouraging
avoidance of wrong corrections at the cost of correcting less errors
overall. The thought that stands behind such emphasis is that a user
would be understanding towards errors he did, of which he is probably
not even aware, not being corrected, but would not be so understanding
when he sees a correction changes what he knows to be correct to a
sentence saying something he did not mean it to. We want to refine
this idea and suggest that there are subtleties we better address
in this intuition. We wish to say there are two different conservatism
types, semantic conservatism and formal one, of which the semantic
is the one we strongly need to adopt and the formal is merely a technicality
for the user. in part 1{*}{*}make sure it is right where all is written,
cross references{*}{*}we address the two conservatism types in part
2 we show semantics can be a consistent and measurable allowing use
of it for ensuring semantic conservatism in part 3 we show current
systems tend to be over too conservative when compared with human
corrections in part 4 we suggest this to be an outcome of the evaluation
measure used, being formally conservative and lacking.

\section{Conservatism - not a single concept}

\subsection{what we really wish to be conservative about}

In the task of grammatical error correction it is important to be
conservative, not to overcorrect. The user expects the minimum corrections
necessary and wants no intervention in what he wishes to say. More
specifically, he expects that what he has said would not be changed
into something he did not. In this we may find two notions of conservatism,
formal conservatism and semantic conservatism. Where any change in
the original string would not be considered formal conservatism, only
changes in meaning are accounted for semantic conservatism. 

In many of the tasks, such as error correction for learner language,
the user does understand his grammar is not perfect and would accept
a change in grammar when needed. Because of this approval we also
hypothesis, and it may call for a user study to prove or disprove
this hypothesis, that users might accept a correct text unit of theirs
to be corrected to another correct text unit with the same meaning.
Maybe even more importantly, we aim to have as many correct sentences
as possible, but as the grammar isn't fully correct in the first place,
nor is the user's understanding of it, failing to correct grammar
is acceptable. Changing meaning will be totally unacceptable, and
also surely detectable by the user. In other words, the users do expect
the corrector to be active and not too formally conservative, but
only as long as it is semantically conservative. 

Moreover, as corrections are based on statistics, they might even
just correct to a more common way of saying the same thing. Such unnecessary
correction is not formally conservative, and at grammatical error
correction maybe be unwanted, but not strictly unwanted as overall
it is semantically conservative still. Additionally, some may even
say this correction is a a needed one having a better grammar considering
Fuzzy Grammar\cite{lakoff1973fuzzy,madnani2011they} or a more fluent
way to say the exact same thing. The latter was suggested as a necessary
shift in the goals of error correction\cite{sakaguchi2016reassessing}
Considering all this, we propose that next generation grammatical
error correctors and evaluation will be focused on semantic conservatism
when possible rather than on formal conservatism.

\section{Semantics in learner language}

\subsection{Uses of semantic annotation}

As semantic annotation was not used before to aid grammatical error
correction, it is worth noting the a-priori reasons for developing
it. The first and perhaps the most obvious use of semantic annotation
would be to use it as a feature for correctors. The annotation may
capture the gist that is supposed to stay the same when correcting,
allowing the corrector to filter results or re-rank them based on
the annotation or just to put it inside the mesh of features and learn
automatically what to do with it, just as done with grammatical annotation.
Later in this section we will not only discuss how to compare those
features, specifically UCCA, but also why it is suspected to be a
valuable feature.

Another approach of using semantic annotation would be for assessment.
Reliable assessment by a gold standard might be hard to obtain (see
\ref{sec:May-lack-of}), and human annotation for each output is great\cite{madnani2011they}
but costly, especially considering development process. In these conditions,
given a reliable semantic annotation we can enhance the reliability
of our assessment. One way to do that might be to decouple the meaning
from the structure. We propose a broad idea for a reduction from grammatical
error detection and a comparable semantics annotation to grammatical
error correction assessment. Lets assume we have both a reliable error
detection tool and a good way to measure semantic changes. Then, we
can transform assessment to a 3 steps assessment. First, detect errors
in the original text. Assess the percentage of needed corrections
that were actually corrected. Second, assess how much of was the semantics
changed. Give a negative score for changing semantics. Third, use
the error detection again to assess how many errors exist in the correction
output, whether uncorrected by the corrector or new errors presented
by the correction process itself. 

This assessment was partially inspired by the WAS evaluation scheme\cite{chodorow2012problems},
in short it states we should account in the assessment for 5 types,
not only the True\textbackslash{}False Positive\textbackslash{}Negative
but also for the case where the annotation calls for a correction,
and the system did a correction, but one that is unlike the annotation's
one. With the proposed assessment we can measure how many of the corrections
were corrected correctly (First + Second), and how many errors do
we have eventually (Third) and combine them to get something similar
to the Precision Recall that is widely used. We can also account for
the places where the error was detected and check if it was corrected
in a way that makes it grammatical and did not change semantics, the
fifth type. We do that without getting a human to confirm this is
indeed a correction.

This system would be even more informative. Allowing assessment of
what exactly is the part in which a corrector failed. Answering questions
like: was it over formally conservative and did not make enough corrections?
Was it making changes in the right places but not correcting grammar
successfully? Was the system correcting grammar but changing things
it was not supposed to? etc.

\subsection{grammar can be annotated but is ill defined}

Syntactic representation is very popular and useful in many NLP tasks{*}{*}cites{*}{*}.
Thus, one thought that rises to mind is to use grammar annotation
to evaluate corrections. While not useless, this approach is not well
defined, and unclear bot practically and theoretically. One would
say that the grammar would be the one induced by the actual words
that appears in the sentence, this would lead to annotation that calls
for applying the syntax of Proper English to the different learner
languages that just don't correspond to it. Thus, the structures may
differ between different learners and they will tell us little about
how to understand the sentence. This approach was being pursued in
\cite{berzak2016universal}. 

Others may suggest, and indeed they have\cite{nagataphrase}, an opposite
approach, saying the grammar meant by the learner is the one we should
tag, but that requires having a corrected form of the sentences. Later\ref{sec:May-lack-of}
we show that for many sentences different corrections are possible.
And where as sentences differ so does their grammar. later in this
section, we propose semantic annotation as a well defined structure.

\subsection{Learner language can be annotated by UCCA}

At least theoretically, semantics are well defined even on ungrammatical
text. With the right tools we might capture at least some of the semantics
of sentences and use them for whatever we wanted grammar for and for
other tasks. In this work we will use Universal Conceptual Cognitive
Annotation (UCCA)\cite{abend2013universal}, we will show that practically
there are semantic annotation schemes that can be used for the purposes
discussed.

But as in the syntactic representation, before we can claim anything
about semantics using UCCA it is needed to show that UCCA is even
consistent when applied to ungrammatical language such as learner
language. To do that we used NUS\cite{dahlmeier2013building} a parallel
corpus of learner language and corrected versions which is the de
facto standard since CoNLL 2013\textbackslash{}14 shared tasks \cite{kao2013conll,ng2014conll}.
The NUS corpus consists of paragraphs of about 400 words each about
various topics. We employed two cognition graduate students, both
with background of working for a couple of years as translator. Each
one had received the guidelines to read and annotated a couple of
proper English paragraphs and then learner language paragraphs as
an exercise. These paragraphs were compared between the annotators
and each disagreement discussed in the hope of finding common annotation
mistakes and choosing a methodological approach to borderline cases.
After that each annotator has annotated 2 learner language paragraphs
consisting of almost 800 UCCA nodes each. Over the uncoordinated paragraphs
we computed the strict inter annotator agreement mentioned by \cite{abend2013universal}
considering each Node in the directed acyclic graph (DAG) of UCCA
annotation as agrees if and only if its label and the labels of all
its span leaves were considered to have the same labels respectively,
from that we derive an F1 score. 

We got an F1 score for the inter annotator agreement of 0.845 with
Precision 0.834 and Recall 0.857 we see that as enough to be a proof
that UCCA can be applied to, especially considering those numbers
are a bit higher than the inter annotator agreement reported in the
reported originally for formal English\cite{abend2013universal}.
We explain the rise in agreement by the fact that the guidelines and
procedures were refined since UCCA was first introduced and not to
superiority of UCCA for annotating learner language. A similar F1
score for inter annotator agreement (0.849) over 2 corrected paragraphs
suggests the same.

\subsection{Semantics are preserved when correcting grammar}

As a next step each annotator annotated corrected paragraphs corresponding
to ones he already annotated, 7 different paragraphs were annotated
in this way. To avoid misleading high score due to the fact that each
annotator annotated both the learner language paragraph and the corrected
paragraph 3 different tuples of paragraphs were annotated by both
annotators allowing a cross comparison, meaning that for each paragraph
we compared the annotation of the learner language done by one annotator
with the annotation of the other annotator done for the corrected
paragraph and vice versa. 

As a next step a comparison between the annotations was needed, but
there exists no measure for how similar two different UCCA annotations
of different texts are. We considered using suggested semantic measures
such as SMATCH\cite{cai2013smatch}but it can not work for UCCA or
DAG similarity measures such as graph kernels (e.g.\cite{kashima2003marginalized}),
but those tend to work on bigger graphs and would be the wrong tool
for the small UCCA DAGs. Thus, a new measure is called upon.

\subsection{Similarity measures\label{subsec:Similarity-measures}}

We propose several new methods to compare UCCA annotation of a learner
language with UCCA annotation of corrected texts, giving a more accurate
measure than the upper bound suggested by \cite{sulem2015conceptual}for
comparing two parallel texts in different languages, while keeping
the essence of comparing how many of the aligned nodes conserve meaning
and tag. For that we may think for a moment on error correction as
translation from learner language to Proper English, and a good translation
would be a translation which keeps the meaning but has the syntax
of English. Considering that, just like in translation we can align
words from the learner language to the corresponding words in English
and keep record of how many of those nodes kept their labels. 

As comparing labels is trivial between a pair,{*}{*}should mention
somewhere weak labeling?{*}{*} we should focus on how we propose to
align nodes. We should note first that alignment should not be at
the token level, as we want to allow tokens to be replaced or removed
as long as the higher structures convey the same meaning. We thus
prune the labels above the leaves, the tokens of the sentence. To
define an alignment of the nodes, we suggest some possible ways, all
based on first aligning the words in order to give order to the DAG
and then comparing the structure in one way or another.

In order to align tokens we use the fact that, unlike in translation,
aligning words is a simple task as most of the words are kept unchanged,
deleted fully, added, or only changed slightly. This allows us to
align words well using edit distance measure, knowing that words that
exists in both sentences will have low edit distance. We consider
aligning to sets of words a bipartite graph matching problem, with
weights according to the edit distance. For tie breaking, we add a
penalty. The penalty is always smaller than 1, the minimum cost of
one action , favoring a sentence order when a word occurs twice.

As to aligning nodes, we can use word spans of each node, based on
the token alignment and the DAG structure, to choose how best to align.
A first and most straightforward approach would be to compare all
pairs of nodes in parallel paragraphs and to each node from one paragraph
assign the one most similar node, span wise from the other. That approach
is quite similar to the inter annotator agreement aligning, but it
has three drawbacks; it is assymetric; it may be over optimistic aligning
nodes without considering the DAG structure; and second it might be
slow for many nodes. Being assymetric is not much of a problem as
we can compute the measure twice and use the mean of the results,
that would also be the case for other assymetric methods we suggest.
In order to address the other drawbacks we propose different aligning
methods.

A second method driven by the assumption that nodes higher in the
hierarchy are more important to the semantic representation is measuring
the largest cut in which nodes are aligned (top down) to each other
and have the same labels. This is expected gives a harsher lower similarity
score but one of which might be more representative of the semantics
that are kept and hopefully more informative for tasks that will use
it.

A third type of methods were token similarity methods, these methods
use one kind of aligning (top down, bottom up or all to all) and only
compare the meaningful nodes. This was called upon in the \cite{sulem2015conceptual}
paper. This approach makes sense due to the fact that some labels
are well defined and thought upon while others are still vague and
call for future work on refining or adjusting them, morover, some
labels are more semantic while other labels are currently just a place
holder as each node must get a level, and the semantic role is not
always clearly defined (e.g. the word ``is'' in ``he is walking''
seem to be more syntactically related than semantically). The unused
labels are center, elaborator, function, relation, linker, ground
and connector.

A bit different way than all the others is to compute the labeled
tree edit distance\cite{zhang1989simple}, for that we first needed
the trees to be ordered, we did that in a top down fashion. An interesting
future work would be to use unordered tree edit distance methods\cite{zhang1992editing}.

All of the code to implement UCCA structures, align them and evaluate
them is also given as a free contribution.{*}{*}{*}link{*}{*}

\subsection{Results}

We present in table {*}{*}{*}{*} the scores of the different presented
methods. For each method we present the average results of 9 tuples
of paragraphs annotated by the same annotator and 6 tuples where each
paragraph was annotated by a different annotator.

Finally, we present as a control measure and a bound on the best score
we can expect to get in such comparison the scores of 7 paragraphs
in which we compare two annotations for the same paragraph using all
the similarity measures discussed, it can be thought of as a different
way to defining inter annotator agreement. Note that a similarity
of 1 and distance of 0 is indeed reached when comparing an annotation
with itself.

In table {*}{*}{*} we present the results of the token analysis, the
upper bound suggested by \cite{sulem2015conceptual}, showing similar
results for learner language - corrected tuples as those seen in English
- French comparison.

\subsection{Discussion}

From the result we learn a number of things, we show that the upper
bounds in table{*}{*} suggest high stability of UCCA over grammatical
error correction, and the results are similar to those shown over
translation. This upper bound seem not to be very strict if the other
measures are to be considered true values, we do note that because
of the aligning errors those measures are actually more of a close
lower bound than an exact value.

We see that measurements for symmetry that are similar to the inter
annotator agreement measure also suggest high stability, achieving
scores not much lower than the one different annotators get for the
same paragraph. This result is quite strong as an inter annotator
agreement is the upper bound being the score of comparing a paragraph
to itself. Most importantly we learn from it all that even when correcting
grammatical errors the semantic structure (as represented by UCCA
at least) is hardly changed and thus can be used as a tool to avoid
introducing semantic changes when trying to only change grammar. The
symmetry measures we introduce can be used to enforce semantic conservatism.
This would be a good place to remind that a direct way to measure
semantic conservatism as we have got here will allow us to be less
formally conservative while focusing on the conservatism users and
hence we are more interested in.

\section{Over conservatism in current error correction attempts}

In recent years a lot of research was done trying to create automatic
error correction\cite{rozovskaya2014building,rozovskaya2010annotating,ng2014conll,kao2013conll}
and given our research on semantic conservatism, it is reasonable
to wonder whether these measures can help improving the existing correctors.
To answer that we need to analyze how conservative these correctors
are, something that we see as insightful and important by itself in
order to improve the correctors.

The first step would be formal conservatism. If corrections are very
formally conservative they are likely to be semantically so too. In
addition, this analysis as will be discussed soon will show that this
analysis is the one really needed at this step of development.

\subsection{Assessing formal conservatism}

Our goal was to analyze the output of all of the participants in Conll
2014 shared task\cite{ng2014conll} and of the current state of the
art \cite{rozovskaya2014building}. We started at manually analyzing,
our impression was that there is a real lack of corrections. Albeit
important, manual analysis is not enough and we aimed for some quantitative
measures. For that we first aligned each learner language text unit
to a corresponding corrected text unit. We used an exact match for
last words in a sentence as a boundary symbol, thus allowing a text
unit to be more than one sentence. This alignment is needed because
we only know the final corrections, a main obstacle that was considered
in the assessment methods as well\cite{dahlmeier2012better}.Our first
result to present will be how many sentences are concatenated and
how many split using the different methods. Moreover, we present the
same measures for the corrections done in the NUS\cite{dahlmeier2013building}.
To have better evaluation of the real goal of corrections we also
compute all of the measures on the TreeBank of Learner Language \cite{berzak2016universal}based
on the Cambridge First Certificate in English (FCE) \cite{yannakoudakis2011new},
a new large parallel corpus containing language of learners native
of different languages.

Next we were interested in how many words are being changed and how
much word order was disrupted. We used the alignment of sentences
and for each sentence we aligned words by edit distance in the same
manner explained in \ref{subsec:Similarity-measures}. We calculated
the number of words changed per sentence to assess how many words
were edited or removed. To measure how much the word order was preserved
we used spearman's rho for the indexes of the aligned words in each
sentence.

\subsection{discussion}

Lets discuss the results in {*}{*}{*}{*} starting at what we see isn't
change in the gold standard. We can call this what calls for formal
conservatism. We see that it is most common for a sentence to have
no change, not to concatenate two consecutive sentences and not to
split a sentence into two. We also observe high correlation coefficient
for most of the sentences. Summing it all together we indeed have
more unchanged than changed in every measure we have. But, with a
closer look we should also notice that it mostly tells us about the
dataset. Specifically the level of English found in the dataset.{*}{*}rephrasing{*}{*}
These measures should be seen as a gold standard of the amount of
corrections to be done, and as we might wish to be a bit conservative
and not exceed it, this is still where we should aim.

When we broaden our view and consider the results of the different
correctors, the picture is clear. All correctors are over conservative.
It is not only that correctors don't tend to overcorrect, they all,
to the last one, by all the measures we checked, undercorrect a lot
and are over conservative. {*}{*}say a word on how do we see that
in each graph and conclude{*}{*}

\begin{figure}
\caption{\protect\includegraphics[width=8cm]{\string"graphs/formal conservatism/differences2_some_competitors\string".pdf}}
text
\end{figure}


\section{May lack of corrections lead to over formal conservatism?\label{sec:May-lack-of}}

\subsection{The idea}

Some corrections might really be too hard for current correctors to
correct, leading to cautious corrections. Another cause for over conservatism
might hide in the assessment methods.before we show it, lets assume
that each ungrammatical sentence has some possible corrections. From
the corrections only 2 are in the gold standard. That would lead to
problems in both assessment and development process. In the assessment,
results will not be reliable, the assessment will only assess whether
we predict the gold standard annotation and not how many of the sentences
where corrected. Perhaps less obvious will be how it affects the development
process. Sentences, even if corrected well, which have more possible
corrections will get lower scores on precision for correcting, while
not getting scores for recall anyway. This will lead either through
machine learning or algorithm development cycles to learn not to correct
those sentences at all. In the rest of this section we will show that
the assumptions we made are more than just assumptions.

\subsection{Formalism}

Lets denote $X=x_{1}\ldots x_{n}\sim World\,Sentences$ as the considered
sentences for evaluation. For each $x_{i}$ there exists $\left\{ y_{1}^{i},\ldots,y_{M}^{i}\right\} \sim x_{i}\,corrections$
the set of gold standard annotations and we consider the output of
the corrector to be a function $f\left(x_{i}\right)$. A certain assessment
statistic is a function $\hat{S}=Eval\left(f\left(x_{1}\right),\left\{ y_{1}^{1},\ldots,y_{M}^{1}\right\} \ldots,f\left(x_{N}\right),\left\{ y_{1}^{N},\ldots,y_{M}^{N}\right\} \right)$.

\subsection{our data}

{*}{*}{*}put it somewhere in context{*}{*}{*}This way of measuring
assumes the sentences do not need context, and while surely untrue
we do assume the context will account to having a bit less possible
corrections but the bigger picture will stay more or less the same.
We also did not use sentences longer than 15 words, assuming those
will be harder to annotate and are more likely to have independent
corrections{*}{*}maybe explain that before?{*}{*}. This choice might
give us a bit lower results in the number of corrections, negating
the effect of the context assumption and only exclaiming the claim
that there are {*}{*}much more corrections than we account for currently{*}{*}

If the NLP community has agreed one correction is not enough{*}{*}cite{*}{*},
we can now say 2 is no magic number either. We can also see what we
lose and gain from different amounts of references, and may also suggest 

reduction + even with indexes the problem is hard as indexes vary
a lot too.

\subsection{Lack of corrections also leads to under estimation of the statistic
and hence over conservatism}

\section{On significance and variation}

While $\mathbb{E}\left(\hat{S}\right)$ vary only as we change $M$
the number of annotations, but not $N$ the number of corrections,
$Var(\hat{S})$ depends on both. We try to assess and give an upper
bound on how much it varies for different $M$ and $N$, allowing
for both a smart allocation of resources when building a corpus and
for assessing on given corpora whether two systems are actually different.

\section{Other things(conclusions? discussion further work?)}

\bibliographystyle{format/acl}
\bibliography{propose}

\end{document}
